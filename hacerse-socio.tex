\SinTikZ

\newpage
\section*{Hacerse socio de la SEMA es muy sencillo, y barato}
%\addcontentsline{toc}{section}{Hacerse socio de la SEMA es muy sencillo, y barato}

La Sociedad Espa�ola de Matem�tica Espa�ola es una sociedad cient�fica totalmente consolidada en el panorama nacional e internacional. En 2016 celebr� el veinticinco aniversario de su fundaci�n. Su historia se describe en su \href{http://www.sema.org.es/es/sociedad/objetivos-e-historia}{p�gina web} y est� documentada en los archivos que se crearon ad hoc con motivo de la conmemoraci�n de su \href{http://www.sema.org.es/es/25-ans-sema}{veinticinco aniversario}.

Ser miembro de la SEMA tiene sus ventajas. Estar�s informado de las distintas actividades de la Matem�tica Aplicada, en el �mbito nacional, e incluso en muchos casos, en el internacional, a trav�s de la edici�n de su Bolet�n electr�nico, del que se publican cuatro n�meros al a�o. La SEMA organiza, con car�cter bienal y en a�os alternos, dos eventos de car�cter internacional: el Congreso de Ecuaciones Diferenciales y Aplicaciones/Congreso de Matem�tica Aplicada, CEDYA/CMA, y
la Escuela Hispano-Francesa Jacques-Louis Lions de Simulaci�n Num�rica en F�sica e Ingenier�a, EHF; en los dos casos, sus socios disfrutan de una cuota reducida, que sumada a la cuota de socio anual alcanzar�a un valor inferior a la cuota de inscripci�n de los no asociados. Adem�s, la SEMA es la responsable de la publicaci�n de la revista SEMA \textit{Journal}, que edita Springer, a la que los socios tienen acceso integral a todos los n�meros y art�culos publicados hasta la fecha. La SEMA otorga dos premios al a�o: el premio SEMA <<Antonio Valle>> al joven investigador, y el premio SEMA al mejor art�culo publicado ese a�o en SEMA \textit{Journal} (siempre que al menos uno de los autores sea miembro de la SEMA). La Sociedad Espa�ola de Matem�tica Aplicada celebra la asamblea anual de socios coincidiendo con la celebraci�n del CEDYA/CMA o de la EHF;  en esta asamblea, sus socios tienen derecho a voz y voto y, entre otras cosas, se elige al presidente de la Sociedad y a los miembros del consejo ejecutivo.

\textbf{Para hacerse socio de la SEMA} basta con rellenar el
\href{https://www.sema.org.es/es/socios/hacerse-socio}{formulario <<\textbf{hazte socio}>>} disponible en la p�gina web de la Sociedad y, a continuaci�n, enviarlo pulsando el bot�n \fbox{Guardar} al final del formulario.
\vspace{.37em}

\textbf{Los estudiantes tienen derecho a una cuota reducida}. Adem�s,
la SEMA mantiene acuerdos de reciprocidad con las sociedades RSME, SIAM, SMAI y SCM con cuotas reducidas para sus socios.

Para los socios de reciprocidad es necesario adjuntar el justificante de miembro de la sociedad que corresponda, y para los estudiantes un certificado de matr�cula del centro.

\vbox to 6cm{\[\resizebox{.925\linewidth}{!}{\begin{tikzpicture}[outline/.style={draw=black!50!white, thick, fill=azulsema, rounded corners}]
\node [outline=red] at (0,1)
{\color{white}
{\large\bf\quad
\begin{tabular}{ll}
{\color{black!20!yellow}\huge\bf Cuotas anuales\strut} &\\[.7em]
 Socio ordinario & 35 \euro\\[.5em]
 Socio estudiante & 17,50 \euro\\[.5em]
 Socio de reciprocidad con la RSME & 14 \euro\\[.5em]
 Socio de reciprocidad con la SIAM & 17,50 \euro \\[.5em]
 Socio de reciprocidad con la SMAI & 17,50 \euro \\[.5em]
 Socio de reciprocidad con la SCM  & 17,50 \euro\qquad\quad\\[.5em]
 Socio extranjero & 35 \euro\\[.5em]
 Socio institucional & 175 \euro\\[.5em]
\end{tabular}}};
\end{tikzpicture}}\]\vss}

\FinSinTikZ

